%%%%%%%%%%%%%%%%%%%%%%%%%%%%%%%%%%%%%%%%%12pt: grandezza carattere
                                        %a4paper: formato a4
                                        %openright: apre i capitoli a destra
                                        %twoside: serve per fare un
                                        %   documento fronteretro
                                        %report: stile tesi (oppure book)
\documentclass[12pt,a4paper,openright,twoside]{report}
%
%%%%%%%%%%%%%%%%%%%%%%%%%%%%%%%%%%%%%%%%%libreria per scrivere in italiano
\usepackage[italian]{babel}
%
%%%%%%%%%%%%%%%%%%%%%%%%%%%%%%%%%%%%%%%%%libreria per accettare i caratteri
                                        %   digitati da tastiera come è à
                                        %   si può usare anche
                                        %   \usepackage[T1]{fontenc}
                                        %   però con questa libreria
                                        %   il tempo di compilazione
                                        %   aumenta
\usepackage[utf8]{inputenc}
%
%%%%%%%%%%%%%%%%%%%%%%%%%%%%%%%%%%%%%%%%%libreria per impostare il documento
\usepackage{fancyhdr}
%
%%%%%%%%%%%%%%%%%%%%%%%%%%%%%%%%%%%%%%%%%libreria per avere l'indentazione
%%%%%%%%%%%%%%%%%%%%%%%%%%%%%%%%%%%%%%%%%   all'inizio dei capitoli, ...
\usepackage{indentfirst}
%
%%%%%%%%%libreria per mostrare le etichette
%\usepackage{showkeys}
%
%%%%%%%%%%%%%%%%%%%%%%%%%%%%%%%%%%%%%%%%%libreria per inserire grafici
\usepackage{graphicx}
%
%%%%%%%%%%%%%%%%%%%%%%%%%%%%%%%%%%%%%%%%%libreria per utilizzare font
                                        %   particolari ad esempio
                                        %   \textsc{}
\usepackage{newlfont}
%
%%%%%%%%%%%%%%%%%%%%%%%%%%%%%%%%%%%%%%%%%librerie matematiche
\usepackage{amssymb}
\usepackage{amsmath}
\usepackage{latexsym}
\usepackage{amsthm}
\usepackage{cite}
\usepackage{listings}
\usepackage{hyperref} 
\usepackage[square,numbers,sort]{natbib} 
%\bibliographystyle{unsrt}
\bibliographystyle{unsrtnat}

\lstset{
	frame=single,
	breaklines=true
}


%
\oddsidemargin=30pt \evensidemargin=20pt%impostano i margini
\hyphenation{sil-la-ba-zio-ne pa-ren-te-si}%serve per la sillabazione: tra parentesi 
					   %vanno inserite come nell'esempio le parole 
%					   %che latex non riesce a tagliare nel modo giusto andando a capo.

%
%%%%%%%%%%%%%%%%%%%%%%%%%%%%%%%%%%%%%%%%%comandi per l'impostazione
                                        %   della pagina, vedi il manuale
                                        %   della libreria fancyhdr
                                        %   per ulteriori delucidazioni
\pagestyle{fancy}\addtolength{\headwidth}{20pt}
\renewcommand{\chaptermark}[1]{\markboth{\thechapter.\ #1}{}}
\renewcommand{\sectionmark}[1]{\markright{\thesection \ #1}{}}
\rhead[\fancyplain{}{\bfseries\leftmark}]{\fancyplain{}{\bfseries\thepage}}
\cfoot{}
%%%%%%%%%%%%%%%%%%%%%%%%%%%%%%%%%%%%%%%%%
\linespread{1.3}                        %comando per impostare l'interlinea
%%%%%%%%%%%%%%%%%%%%%%%%%%%%%%%%%%%%%%%%%definisce nuovi comandi
%

%%%%%%%%%%%%%%%%%%%%%%%%%%%%%%%%%%%%%%
% Comandi Custom %
%%%%%%%%%%%%%%%%%%%%%%%%%%%%%%%%%%%%%%
\newcommand{\xstudent}{Falconi Eleonora}
\newcommand{\xsupervisor}{Silvia Mirri}
\newcommand{\xcorrelatore}{Giovanni Delnevo}



%%%%%%%%%%%%%%%%%%%%%%%%%%%%%%%%%%%%%%
% Fine Preambolo %
% Inizio documento%
%%%%%%%%%%%%%%%%%%%%%%%%%%%%%%%%%%%%%%


\begin{document}

	%%%%%%%%%%%%%%%%%%%%%%%%%%%%%%%%%%%%%%%%
	% Scelta delle dimensioni della pagina %
	%%%%%%%%%%%%%%%%%%%%%%%%%%%%%%%%%%%%%%%%

	%\setlength{\textwidth}{13.5cm}
	%\setlength{\textheight}{19cm}
	%\setlength{\footskip}{3cm}
	
	%%%%%%%%%%%%%%%%%%%%%%%%%
	% inizio prefazione
	%
	% pagina del titolo, indice, sommario
	%%%%%%%%%%%%%%%%%%%%%%%%%
	
	
%\textwidth=450pt
\oddsidemargin=25pt

\begin{titlepage}
\begin{center}
{{\Large{\textsc{Alma Mater Studiorum}}}\\
{\Large{\textsc{Universit\`a di Bologna}}} \\
{\textsc{Campus di Cesena}} \rule[0.1cm]{14cm}{0.1mm}
		\rule[0.5cm]{14cm}{0.6mm}
DIPARTIMENTO DI INFORMATICA – SCIENZA E INGEGNERIA
Corso di Laurea in Ingegneria e Scienze Informatiche }
\end{center}
\vspace{15mm}
\begin{center}
{\LARGE{\bf Comparazione di algoritmi di machine learning tradizionali}}\\
\vspace{3mm}
{\LARGE{\bf per il riconoscimento delle emozioni vocali}}\\
\end{center}
\vspace{40mm}
\par
\noindent
\begin{minipage}[t]{0.47\textwidth}
{\large{\bf Relatore:\\
Prof.ssa \xsupervisor}}
\vspace{5mm}
{\large{\bf \\Correlatore:\\
Dr. \xcorrelatore}}
\end{minipage}
\hfill
\begin{minipage}[t]{0.47\textwidth}\raggedleft
{\large{\bf Presentata da:\\
\xstudent}}
\end{minipage}
\vspace{20mm}
\begin{center}
{\large{\bf Sessione II\\%inserire il numero della sessione in cui ci si laurea
Anno Accademico 2023-2024}}%inserire l'anno accademico a cui si è iscritti
\end{center}
\end{titlepage}



	\begin{titlepage}                       %crea un ambiente libero da vincoli
                                        %   di margini e grandezza caratteri:
                                        %   si pu\`o modificare quello che si
                                        %   vuole, tanto fuori da questo
                                        %   ambiente tutto viene ristabilito
%
\thispagestyle{empty}                   %elimina il numero della pagina
\topmargin=6.5cm                        %imposta il margina superiore a 6.5cm
\raggedleft                             %incolonna la scrittura a destra
\large                                  %aumenta la grandezza del carattere
                                        %   a 14pt
\em                                     %emfatizza (corsivo) il carattere
Questa \`e la \textsc{Dedica}:\\
ognuno pu\`o scrivere quello che vuole, \\
anche nulla \ldots                      %\ldots lascia tre puntini
\newpage                                %va in una pagina nuova

%
%%%%%%%%%%%%%%%%%%%%%%%%%%%%%%%%%%%%%%%%
\clearpage{\pagestyle{empty}\cleardoublepage}%non numera l'ultima pagina sinistra
\end{titlepage}
    
    %
%%%%%%%%%%%%%%%%%%%%%%%%%%%%%%%%%%%%%%%%
\pagenumbering{roman}                   %serve per mettere i numeri romani
\chapter*{Introduzione}                 %crea l'introduzione (un capitolo
                                        %   non numerato)
%%%%%%%%%%%%%%%%%%%%%%%%%%%%%%%%%%%%%%%%%imposta l'intestazione di pagina
\rhead[\fancyplain{}{\bfseries
INTRODUZIONE}]{\fancyplain{}{\bfseries\thepage}}
\lhead[\fancyplain{}{\bfseries\thepage}]{\fancyplain{}{\bfseries
INTRODUZIONE}}
%%%%%%%%%%%%%%%%%%%%%%%%%%%%%%%%%%%%%%%%%aggiunge la voce Introduzione
                                        %   nell'indice
\addcontentsline{toc}{chapter}{Introduzione}
Questa \`e l'introduzione.
%%%%%%%%%%%%%%%%%%%%%%%%%%%%%%%%%%%%%%%%%non numera l'ultima pagina sinistra
\clearpage{\pagestyle{empty}\cleardoublepage}

\tableofcontents                        %crea l'indice
%%%%%%%%%%%%%%%%%%%%%%%%%%%%%%%%%%%%%%%%%imposta l'intestazione di pagina
\rhead[\fancyplain{}{\bfseries\leftmark}]{\fancyplain{}{\bfseries\thepage}}
\lhead[\fancyplain{}{\bfseries\thepage}]{\fancyplain{}{\bfseries
INDICE}}
%%%%%%%%%%%%%%%%%%%%%%%%%%%%%%%%%%%%%%%%%non numera l'ultima pagina sinistra
\clearpage{\pagestyle{empty}\cleardoublepage}
\listoffigures                          %crea l'elenco delle figure
%%%%%%%%%%%%%%%%%%%%%%%%%%%%%%%%%%%%%%%%%non numera l'ultima pagina sinistra
\clearpage{\pagestyle{empty}\cleardoublepage}
\listoftables                           %crea l'elenco delle tabelle
%%%%%%%%%%%%%%%%%%%%%%%%%%%%%%%%%%%%%%%%%non numera l'ultima pagina sinistra
    
    

	
	%%%%%%%%%%%%%%%%%%%%%%%%%
	% inizio corpo del documento
	%
	% sequenze delle varie sezioni
	% è consigliato mantenere una struttura logica ben definita per separare le sezioni
	% si consiglia di reificare tale struttura fisicamente sul file system
	%%%%%%%%%%%%%%%%%%%%%%%%%
	

	
	% inclusione delle sezioni
	\clearpage{\pagestyle{empty}\cleardoublepage}
\chapter{Contesto applicativo}                %crea il capitolo
%%%%%%%%%%%%%%%%%%%%%%%%%%%%%%%%%%%%%%%%%imposta l'intestazione di pagina
\lhead[\fancyplain{}{\bfseries\thepage}]{\fancyplain{}{\bfseries\rightmark}}
\pagenumbering{arabic}                  %mette i numeri arabi

In questo capitolo vengono presentati i fondamenti teorici e le ragioni che hanno guidato lo sviluppo di questa tesi, fornendo una visione d'insieme sulle principali tappe del percorso di ricerca. Viene inoltre esaminato il ruolo delle innovazioni tecnologiche per quanto riguarda l'intelligenza artificiale e gli algoritmi di Machine learning, che hanno permesso a macchine e sistemi di interpretare le emozioni espresse attraverso la voce~\cite{9383000}.

%Questo \`e il primo capitolo di una tesi scritta in Latex ~\cite{latex}
%\section{Intelligenza Artificiale}                 %crea la sezione
%Questa \`e la prima sezione.

%Ora vediamo un elenco numerato:         %crea un elenco numerato
%\begin{enumerate}
%\item primo oggetto
%\item secondo oggetto
%\item terzo oggetto
%\item quarto oggetto
%\end{enumerate}

%\begin{figure}[h]                       %crea l'ambiente figura; [h] sta
                                        %   per here, cioè la figura va qui
%\begin{center}                          %centra nel mezzo della pagina
                                        %   la figura
%\includegraphics[width=5cm]{figura.eps}%inserisce una figura larga 5cm
                                        %se si vuole usare va scommentata
%
%%%%%%%%%%%%%%%%%%%%%%%%%%%%%%%%%%%%%%%%%inserisce la legenda ed etichetta
                                        %   la figura con \label{fig:prima}
%\caption[legenda elenco figure]{legenda sotto la figura}\label{fig:prima}
%\end{center}
%\end{figure}

\section{Cos è lo Speech Emotion Recognition?}

Il linguaggio parlato è uno dei mezzi più significativi e naturali attraverso cui gli esseri umani comunicano le proprie emozioni, stati cognitivi e intenzioni. Negli ultimi decenni, il riconoscimento delle emozioni nel parlato ha attirato molta attenzione nell'ambito dell'informatica incentrata sull'uomo, poiché rappresenta un aspetto fondamentale per comprendere il comportamento emotivo umano. Le crescenti applicazioni del SER, in particolare quelle legate all'interazione uomo-computer e al "computing" affettivo, lo rendono un componente centrale della prossima generazione di sistemi informatici, dove un'interfaccia naturale tra uomo e macchina consentirà l'automazione di servizi che richiedono una buona comprensione dello stato emotivo dell'utente~\cite{6913013}.

 I sistemi di riconoscimento delle emozioni nel parlato sono metodologie che analizzano e classificano i segnali vocali per identificare le emozioni in essi presenti. 
Lo Speech Emotion Recognition (SER) è un campo di studio che si occupa di dedurre le emozioni umane dai segnali vocali. Questi sistemi, spesso considerati problemi di classificazione o regressione, si concentrano sull'identificazione dell'input vocale come appartenente a diverse categorie emotive~\cite{GEORGE2024127015}.


\subsection{Definizione e Origine del SER}
\subsection{Importanza e Applicazione}
Ora vediamo un elenco puntato:
\begin{itemize}                         %crea un elenco puntato
\item primo oggetto
\item secondo oggetto
\end{itemize}

\section{Le Emozioni nel Parlato} %Elementi Chiave del Riconoscimento delle Emozioni nel parlato

\subsection{Modelli Psicologici delle emozioni}
modello di Ekman, valenza-arousal) e il loro impatto sull'annotazione e il riconoscimento delle emozioni nel parlato
\subsection{Categorizzazione e tassonomie}
Descrizione delle varie classificazioni delle emozioni usate nel SER (ad esempio, emozioni discrete vs. emozioni continue).
\subsection{Caratteristiche Vocali e Parametri}
Analisi dei parametri vocali utilizzati nel riconoscimento delle emozioni, come il tono, la velocità, l'intensità e la prosodia.
\subsection{Impatto della variabilità linguistica e culturale}

\section{Tecniche di estrazione delle Feature}
\subsection{Feature Acustiche}
Dettagli sulle principali caratteristiche acustiche utilizzate nel SER (ad esempio, MFCC, Chroma, ecc.).
\subsection{Feature Prosodiche}

\subsection{Approcci di Estrazione Basati su Deep Learning}
Discussione sulle tecniche moderne come l'uso di convolutive neural networks (CNN) per l'estrazione automatica delle feature.

Vediamo un elenco descrittivo:
\begin{description}                     %crea un elenco descrittivo
  \item[OGGETTO1] prima descrizione;
  \item[OGGETTO2] seconda descrizione;
  \item[OGGETTO3] terza descrizione.
\end{description}

%%%%%%%%%%%%%%%%%%%%%%%%%%%%%%%%%%%%%%%%%crea una sottosezione
%\subsection{Altra SottoSezione}
%%%%%%%%%%%%%%%%%%%%%%%%%%%%%%%%%%%%%%%%%crea una sottosottosezione
%\subsubsection{SottoSottoSezione}Questa sottosottosezione non viene
numerata, ma \`e solo scritta in grassetto.
\section{Raccolta Dati e Annotazione nel SER}                 %crea una sottosezione
\subsection{Dataset Utilizzati}
Descrizione dei dataset più utilizzati come IEMOCAP, RAVDESS, CREMA-D, con analisi delle loro caratteristiche.
\subsection{Metodi di Annotazione delle Emozioni}
Esplorazione dei metodi di annotazione dei dati e delle problematiche legate alla soggettività nel labeling delle emozioni.
\subsection{Sfide principali}
Discussione sulle difficoltà nell'annotare correttamente le emozioni nei dati audio, soprattutto riguardo alla soggettività e alla variabilità culturale.
\subsubsection{}
Vediamo la creazione di una tabella; la tabella \ref{tab:uno}
(richiamo il nome della tabella utilizzando la label che ho messo sotto):
la facciamo di tre righe e tre colonne, la prima colonna
``incolonnata'' a destra (r) e le altre centrate (c):\\
\begin{table}[h]                        %ambiente tabella
                                        %(serve per avere la legenda)
\begin{center}                          %centra nella pagina la tabella
\begin{tabular}{r|c|c}                  %tre colonne con righe verticali
                                        %   prodotte con |
\hline \hline                           %inserisce due righe orizzontali
$(1,1)$ & $(1,2)$ & $(1,3)$\\           %& separa le colonne e con
\hline                                  %inserisce una riga orizzontale
$(2,1)$ & $(2,2)$ & $(2,3)$\\           %  \\ va a capo
\hline                                  %inserisce una riga orizzontale
$(3,1)$ & $(3,2)$ & $(3,3)$\\
\hline \hline                           %inserisce due righe orizzontali
\end{tabular}
\caption[legenda elenco tabelle]{legenda tabella}\label{tab:uno}
\end{center}
\end{table}
%\section{Altra Sezione}\label{sec:prova}%posso mettere le label anche
                                        %   alle section
%\subsection{Listati dei programmi}
%\subsubsection{Primo Listato}
\begin{verbatim}
        In questo ambiente     posso scrivere      come voglio,
lasciare gli spazi che voglio e non % commentare quando voglio
e ci sarà scritto tutto.
Quando lo uso è meglio che disattivi il Wrap del WinEdt
\end{verbatim}
	\clearpage{\pagestyle{empty}\cleardoublepage}
\chapter{Le tecnologie utilizzate}                %crea il capitolo
%%%%%%%%%%%%%%%%%%%%%%%%%%%%%%%%%%%%%%%%%imposta l'intestazione di pagina

Questo \`e il secondo capitolo.
\section{Machine Learning}                 %crea la sezione
Questa \`e la prima sezione.



\section{Alberi di Decisione e Metodi Ensemble}                 %crea la sezione
Questa \`e la seconda sezione.
\subsection{Random forest}
\subsection{Gradient Boosting}
\subsection{Random forest}
\subsection{Extreme Gradient Boosting(XGB)}

\section{Modelli lineari e Neurali}                 %crea la sezione
Questa \`e la terza sezione.
\subsection{Logistic Regression}
\subsection{Multi-layer Perceptron classifier(MLP)}
\subsection{C-Support Vector Classification(SVC)}


	\clearpage{\pagestyle{empty}\cleardoublepage}
\chapter{Terzo capitolo}                %crea il capitolo
%%%%%%%%%%%%%%%%%%%%%%%%%%%%%%%%%%%%%%%%%imposta l'intestazione di pagina

Questo \`e il terzo capitolo.

\section{Prima Sezione}                 %crea la sezione
Questa \`e la prima sezione.

\newpage

\section{Seconda Sezione}                 %crea la sezione
Questa \`e la seconda sezione.

\newpage

\section{Terza Sezione}                 %crea la sezione
Questa \`e la terza sezione.

	

	%%%%%%%%%%%%%%%%%%%%%%%%%
	% inizio parte finale del documento
	%
	% eventuali appendici, bibliografia obbligatoria,
	% eventuale lista delle tabelle e delle figure (nel caso decommentare 
	% la riga con i comandi \listoffigures e \listoftables)
	%%%%%%%%%%%%%%%%%%%%%%%%%
	
    %%%%%%%%%%%%%%%%%%%%%%%%%%%%%%%%%%%%%%%%%non numera l'ultima pagina sinistra
\clearpage{\pagestyle{empty}\cleardoublepage}
%%%%%%%%%%%%%%%%%%%%%%%%%%%%%%%%%%%%%%%%%per fare le conclusioni
\chapter*{Conclusioni}
%%%%%%%%%%%%%%%%%%%%%%%%%%%%%%%%%%%%%%%%%imposta l'intestazione di pagina
\rhead[\fancyplain{}{\bfseries
CONCLUSIONI}]{\fancyplain{}{\bfseries\thepage}}
\lhead[\fancyplain{}{\bfseries\thepage}]{\fancyplain{}{\bfseries
CONCLUSIONI}}
%%%%%%%%%%%%%%%%%%%%%%%%%%%%%%%%%%%%%%%%%aggiunge la voce Conclusioni
                                        %   nell'indice
\addcontentsline{toc}{chapter}{Conclusioni} Queste sono le
conclusioni.\\

Lorem ipsum dolor sit amet, consectetur adipiscing elit. Quisque a magna quis nunc venenatis vestibulum. Curabitur commodo efficitur ipsum, non ullamcorper tellus. Duis dictum commodo nisi nec venenatis. Donec euismod pulvinar finibus. Suspendisse lorem mi, suscipit quis faucibus ut, luctus in justo. Cras pulvinar arcu ut ullamcorper pulvinar. Aliquam dictum tortor quis diam luctus, quis tristique tortor ultrices. Integer et lacus a velit efficitur convallis. Morbi enim erat, fermentum vel nulla id, viverra vehicula nisi. Integer non auctor leo, eu convallis massa. Cras eu cursus ligula. Nunc non purus et sem vehicula viverra ut nec nibh.

Quisque posuere purus quis eros auctor efficitur. Etiam mattis vitae nulla et blandit. Nulla a orci magna. Cras ac elit enim. Vestibulum nec nisl metus. Mauris congue velit nec malesuada scelerisque. Sed dignissim, enim vitae semper fermentum, mauris leo vestibulum nisl, in malesuada nibh felis nec dui.

Nullam sit amet tellus eget mi varius commodo. Vestibulum sit amet egestas odio. Nam in ullamcorper quam, nec efficitur augue. Curabitur eget elit in leo eleifend tempor vel lobortis lorem. Duis neque dui, tempus eu sollicitudin ac, lobortis sit amet odio. Morbi eleifend, tellus a varius consequat, enim erat sagittis justo, ac rutrum ipsum augue in leo. Suspendisse non mi ante.

Praesent sed pretium dui, id volutpat tortor. Suspendisse tortor lorem, vestibulum vitae ullamcorper vitae, tincidunt nec leo. Proin interdum congue blandit. Ut bibendum sagittis leo, nec venenatis urna mollis id. Donec nec erat non justo maximus venenatis. In mollis elit eu odio maximus porta. Vestibulum varius turpis sit amet orci blandit, vitae volutpat erat viverra.

Suspendisse nunc urna, elementum ut purus a, sagittis porta velit. Integer ultricies convallis tortor id pellentesque. Duis et sem a mi bibendum congue. Morbi ut tellus cursus, laoreet ipsum rutrum, condimentum felis. Proin velit mi, ultricies a urna nec, facilisis pretium mi. Pellentesque tristique interdum purus, a facilisis mi tempor quis. Sed finibus venenatis ligula porttitor porttitor. Suspendisse cursus lorem nec velit commodo fringilla.
Lorem ipsum dolor sit amet, consectetur adipiscing elit. Quisque a magna quis nunc venenatis vestibulum. Curabitur commodo efficitur ipsum, non ullamcorper tellus. Duis dictum commodo nisi nec venenatis. Donec euismod pulvinar finibus. Suspendisse lorem mi, suscipit quis faucibus ut, luctus in justo. Cras pulvinar arcu ut ullamcorper pulvinar. Aliquam dictum tortor quis diam luctus, quis tristique tortor ultrices. Integer et lacus a velit efficitur convallis. Morbi enim erat, fermentum vel nulla id, viverra vehicula nisi. Integer non auctor leo, eu convallis massa. Cras eu cursus ligula. Nunc non purus et sem vehicula viverra ut nec nibh.

Quisque posuere purus quis eros auctor efficitur. Etiam mattis vitae nulla et blandit. Nulla a orci magna. Cras ac elit enim. Vestibulum nec nisl metus. Mauris congue velit nec malesuada scelerisque. Sed dignissim, enim vitae semper fermentum, mauris leo vestibulum nisl, in malesuada nibh felis nec dui.

Nullam sit amet tellus eget mi varius commodo. Vestibulum sit amet egestas odio. Nam in ullamcorper quam, nec efficitur augue. Curabitur eget elit in leo eleifend tempor vel lobortis lorem. Duis neque dui, tempus eu sollicitudin ac, lobortis sit amet odio. Morbi eleifend, tellus a varius consequat, enim erat sagittis justo, ac rutrum ipsum augue in leo. Suspendisse non mi ante.

Praesent sed pretium dui, id volutpat tortor. Suspendisse tortor lorem, vestibulum vitae ullamcorper vitae, tincidunt nec leo. Proin interdum congue blandit. Ut bibendum sagittis leo, nec venenatis urna mollis id. Donec nec erat non justo maximus venenatis. In mollis elit eu odio maximus porta. Vestibulum varius turpis sit amet orci blandit, vitae volutpat erat viverra.

Suspendisse nunc urna, elementum ut purus a, sagittis porta velit. Integer ultricies convallis tortor id pellentesque. Duis et sem a mi bibendum congue. Morbi ut tellus cursus, laoreet ipsum rutrum, condimentum felis. Proin velit mi, ultricies a urna nec, facilisis pretium mi. Pellentesque tristique interdum purus, a facilisis mi tempor quis. Sed finibus venenatis ligula porttitor porttitor. Suspendisse cursus lorem nec velit commodo fringilla.
    %\clearpage{\pagestyle{empty}\cleardoublepage}

\cleardoublepage

\rhead[\fancyplain{}{\bfseries BIBLIOGRAFIA}]{\fancyplain{}{\bfseries\thepage}}
\lhead[\fancyplain{}{\bfseries\thepage}]{\fancyplain{}{\bfseries BIBLIOGRAFIA}}
%%%%%%%%%%%%%%%%%%%%%%%%%%%%%%%%%%%%%%%%% aggiunge l'intestazione di pagina
%\chapter*{Bibliografia}


\phantomsection


\addcontentsline{toc}{chapter}{Bibliografia}
%%%%%%%%%%%%%%%%%%%%%%%%%%%%%%%%%%%%%%%%% aggiunge la voce Bibliografia nell'indice





\bibliography{backMatter/biblio}{}
\bibliographystyle{plain}


    \rhead[\fancyplain{}{\bfseries \leftmark}]{\fancyplain{}{\bfseries
\thepage}}
%%%%%%%%%%%%%%%%%%%%%%%%%%%%%%%%%%%%%%%%%aggiunge la voce Bibliografia
                                        %   nell'indice

%%%%%%%%%%%%%%%%%%%%%%%%%%%%%%%%%%%%%%%%%non numera l'ultima pagina sinistra
\clearpage{\pagestyle{empty}\cleardoublepage}
\chapter*{Ringraziamenti}
\thispagestyle{empty}

\addcontentsline{toc}{chapter}{Ringraziamenti}

Qui possiamo ringraziare il mondo intero!!!!!!!!!!\\
Ovviamente solo se uno vuole, non \`e obbligatorio.
	
		
	%\input{./Appendice/appendice.tex}
	%%\clearpage{\pagestyle{empty}\cleardoublepage}

\cleardoublepage

\rhead[\fancyplain{}{\bfseries BIBLIOGRAFIA}]{\fancyplain{}{\bfseries\thepage}}
\lhead[\fancyplain{}{\bfseries\thepage}]{\fancyplain{}{\bfseries BIBLIOGRAFIA}}
%%%%%%%%%%%%%%%%%%%%%%%%%%%%%%%%%%%%%%%%% aggiunge l'intestazione di pagina
%\chapter*{Bibliografia}


\phantomsection


\addcontentsline{toc}{chapter}{Bibliografia}
%%%%%%%%%%%%%%%%%%%%%%%%%%%%%%%%%%%%%%%%% aggiunge la voce Bibliografia nell'indice





\bibliography{backMatter/biblio}{}
\bibliographystyle{plain}


	
	\nocite{*}

	%\cleardoublepage
	%\addcontentsline{toc}{chapter}{Bibliografia}


	
	%\listoffigures
	%\listoftables


\end{document}
