\clearpage{\pagestyle{empty}\cleardoublepage}
\chapter{Contesto applicativo}                %crea il capitolo
%%%%%%%%%%%%%%%%%%%%%%%%%%%%%%%%%%%%%%%%%imposta l'intestazione di pagina
\lhead[\fancyplain{}{\bfseries\thepage}]{\fancyplain{}{\bfseries\rightmark}}
\pagenumbering{arabic}                  %mette i numeri arabi

In questo capitolo vengono presentati i fondamenti teorici e le ragioni che hanno guidato lo sviluppo di questa tesi, fornendo una visione d'insieme sulle principali tappe del percorso di ricerca. Viene inoltre esaminato il ruolo delle innovazioni tecnologiche per quanto riguarda l'intelligenza artificiale e gli algoritmi di Machine learning, che hanno permesso a macchine e sistemi di interpretare le emozioni espresse attraverso la voce~\cite{9383000}.

%Questo \`e il primo capitolo di una tesi scritta in Latex ~\cite{latex}
%\section{Intelligenza Artificiale}                 %crea la sezione
%Questa \`e la prima sezione.

%Ora vediamo un elenco numerato:         %crea un elenco numerato
%\begin{enumerate}
%\item primo oggetto
%\item secondo oggetto
%\item terzo oggetto
%\item quarto oggetto
%\end{enumerate}

%\begin{figure}[h]                       %crea l'ambiente figura; [h] sta
                                        %   per here, cioè la figura va qui
%\begin{center}                          %centra nel mezzo della pagina
                                        %   la figura
%\includegraphics[width=5cm]{figura.eps}%inserisce una figura larga 5cm
                                        %se si vuole usare va scommentata
%
%%%%%%%%%%%%%%%%%%%%%%%%%%%%%%%%%%%%%%%%%inserisce la legenda ed etichetta
                                        %   la figura con \label{fig:prima}
%\caption[legenda elenco figure]{legenda sotto la figura}\label{fig:prima}
%\end{center}
%\end{figure}

\section{Cos è lo Speech Emotion Recognition?}

Il linguaggio parlato è uno dei mezzi più significativi e naturali attraverso cui gli esseri umani comunicano le proprie emozioni, stati cognitivi e intenzioni. Negli ultimi decenni, il riconoscimento delle emozioni nel parlato ha attirato molta attenzione nell'ambito dell'informatica incentrata sull'uomo, poiché rappresenta un aspetto fondamentale per comprendere il comportamento emotivo umano. Le crescenti applicazioni del SER, in particolare quelle legate all'interazione uomo-computer e al "computing" affettivo, lo rendono un componente centrale della prossima generazione di sistemi informatici, dove un'interfaccia naturale tra uomo e macchina consentirà l'automazione di servizi che richiedono una buona comprensione dello stato emotivo dell'utente~\cite{6913013}.

 I sistemi di riconoscimento delle emozioni nel parlato sono metodologie che analizzano e classificano i segnali vocali per identificare le emozioni in essi presenti. 
Lo Speech Emotion Recognition (SER) è un campo di studio che si occupa di dedurre le emozioni umane dai segnali vocali. Questi sistemi, spesso considerati problemi di classificazione o regressione, si concentrano sull'identificazione dell'input vocale come appartenente a diverse categorie emotive~\cite{GEORGE2024127015}.


\subsection{Definizione e Origine del SER}
\subsection{Importanza e Applicazione}
Ora vediamo un elenco puntato:
\begin{itemize}                         %crea un elenco puntato
\item primo oggetto
\item secondo oggetto
\end{itemize}

\section{Le Emozioni nel Parlato} %Elementi Chiave del Riconoscimento delle Emozioni nel parlato

\subsection{Modelli Psicologici delle emozioni}
modello di Ekman, valenza-arousal) e il loro impatto sull'annotazione e il riconoscimento delle emozioni nel parlato
\subsection{Categorizzazione e tassonomie}
Descrizione delle varie classificazioni delle emozioni usate nel SER (ad esempio, emozioni discrete vs. emozioni continue).
\subsection{Caratteristiche Vocali e Parametri}
Analisi dei parametri vocali utilizzati nel riconoscimento delle emozioni, come il tono, la velocità, l'intensità e la prosodia.
\subsection{Impatto della variabilità linguistica e culturale}

\section{Tecniche di estrazione delle Feature}
\subsection{Feature Acustiche}
Dettagli sulle principali caratteristiche acustiche utilizzate nel SER (ad esempio, MFCC, Chroma, ecc.).
\subsection{Feature Prosodiche}

\subsection{Approcci di Estrazione Basati su Deep Learning}
Discussione sulle tecniche moderne come l'uso di convolutive neural networks (CNN) per l'estrazione automatica delle feature.

Vediamo un elenco descrittivo:
\begin{description}                     %crea un elenco descrittivo
  \item[OGGETTO1] prima descrizione;
  \item[OGGETTO2] seconda descrizione;
  \item[OGGETTO3] terza descrizione.
\end{description}

%%%%%%%%%%%%%%%%%%%%%%%%%%%%%%%%%%%%%%%%%crea una sottosezione
%\subsection{Altra SottoSezione}
%%%%%%%%%%%%%%%%%%%%%%%%%%%%%%%%%%%%%%%%%crea una sottosottosezione
%\subsubsection{SottoSottoSezione}Questa sottosottosezione non viene
numerata, ma \`e solo scritta in grassetto.
\section{Raccolta Dati e Annotazione nel SER}                 %crea una sottosezione
\subsection{Dataset Utilizzati}
Descrizione dei dataset più utilizzati come IEMOCAP, RAVDESS, CREMA-D, con analisi delle loro caratteristiche.
\subsection{Metodi di Annotazione delle Emozioni}
Esplorazione dei metodi di annotazione dei dati e delle problematiche legate alla soggettività nel labeling delle emozioni.
\subsection{Sfide principali}
Discussione sulle difficoltà nell'annotare correttamente le emozioni nei dati audio, soprattutto riguardo alla soggettività e alla variabilità culturale.
\subsubsection{}
Vediamo la creazione di una tabella; la tabella \ref{tab:uno}
(richiamo il nome della tabella utilizzando la label che ho messo sotto):
la facciamo di tre righe e tre colonne, la prima colonna
``incolonnata'' a destra (r) e le altre centrate (c):\\
\begin{table}[h]                        %ambiente tabella
                                        %(serve per avere la legenda)
\begin{center}                          %centra nella pagina la tabella
\begin{tabular}{r|c|c}                  %tre colonne con righe verticali
                                        %   prodotte con |
\hline \hline                           %inserisce due righe orizzontali
$(1,1)$ & $(1,2)$ & $(1,3)$\\           %& separa le colonne e con
\hline                                  %inserisce una riga orizzontale
$(2,1)$ & $(2,2)$ & $(2,3)$\\           %  \\ va a capo
\hline                                  %inserisce una riga orizzontale
$(3,1)$ & $(3,2)$ & $(3,3)$\\
\hline \hline                           %inserisce due righe orizzontali
\end{tabular}
\caption[legenda elenco tabelle]{legenda tabella}\label{tab:uno}
\end{center}
\end{table}
%\section{Altra Sezione}\label{sec:prova}%posso mettere le label anche
                                        %   alle section
%\subsection{Listati dei programmi}
%\subsubsection{Primo Listato}
\begin{verbatim}
        In questo ambiente     posso scrivere      come voglio,
lasciare gli spazi che voglio e non % commentare quando voglio
e ci sarà scritto tutto.
Quando lo uso è meglio che disattivi il Wrap del WinEdt
\end{verbatim}